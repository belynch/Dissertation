\chapter{Conclusion}

  \section{Overview}
  In this thesis I have provided the design for two protocols that make use
  of AODV's route discovery mechanism to perform attendance tracking. The first
  protocol \textit{Roll Call}, is designed for lecture theatre-like scenarios in
  which the attendance of a pre-determined list of users is being taken. The
  second protocol \textit{Route to zero}, is designed for scenarios in which
  the attendance of a group of possibly unknown users is being taken.

  An AODV module with route discovery functionality was implemented for use in
  Bluetooth Low Energy. In addition to this, the Roll Call and Route-to-Zero protocols were implemented
  using the AODV module and tested in a small to medium scale network of nRF51
  devices. The results of the evaluation show the protocol working with three
  hops in the network, but further fine tuning of the BLE setting is required
  to improve the performance of the protocol.

  \section{Discussion}
  The two designs were capable of performing node discovery, but it is difficult
  to say how successful they were without any other references. Based on the recorded
  data, it would take the RTZ protocol 90 seconds to take the attendance in a lecture
  theatre of 120 students. This would be further increased in networks with mobility
  and in areas with high interference - an aspect of networks untested in this thesis.

  The AODV discovery mechanism was suitable for the task but in both designs, the amount
  of RREQ packets flooded through the system was quite high.
  It is possible that the metrics measured could be further improved by reducing
  the advertisement period and increasing the scanning interval, allowing for
  packets to be flooded through the network at a faster pace. It would also be necessary
  to experiment in larger networks - either with physical devices, or through
  simulation.

  Overall, AODV seems to have been a good choice. It outperformed the other
  investigated protocols in mobile networks, which could be desirable in attendance
  tracking applications, but was not tested in this thesis. For an approach
  focused on purely static networks, RPL or OLSR, would likely be better choices,
  but have a higher implementation complexity and their performance in BLE is
  difficult to predict.

  \section{Future Work}
  \subsection{Improvements to AODV}
  There are a number of improvements that could be made to the AODV module, namely
  the implementation of route maintenance. This would allow for attendance tracking
  in scenarios of high mobility and would be necessary in larger networks were there
  is a much higher chance of route failure.

  \subsection{Simulation}
  The testing of the designs outlined in this thesis was only performed in a small
  ten node network. It is difficult to test in larger physical networks as the devices
  have to be acquired, programmed, and space has to be located. With a minimum range
  of 4.32m, a very large space would be required to test networks with hop counts
  in the double digits and higher. One solution to this which is commonly used, is
  a simulation environment that replicates these larger networks which provides the
  advantage of not requiring physical devices or space.

  One issue with this is that there are few BLE simulation tools. There is a
  BLE simulator \cite{ble_simulator} implemented in the OMNeT++ simulator \cite{omnet}
  but there are no instructions, no updates, and it has quite poor code quality as
  it was initially intended to quickly check some of the author's ideas. As a
  result of this, in the early stages of this thesis, I decided to implement
  a higher level BLE simulator, which accurately simulated BLE channels in the
  physical layer, and advertising and scanning. The implementation was written
  in Java and drew from the OMNeT++ simulator. Unfortunately, there wasn't sufficient
  time to implement this as well as implementing the designs on physical hardware
  and so simulation development was halted to prioritise the physical implementation.

  \subsection{Implementation of other protocols}
  It would be interesting to implement some of the other protocols discussed in
  chapter \ref{background}, especially RPL. RPL outperformed AODV in
  section \ref{performance} so it would be interesting to see its performance
  when used for attendance tracking. The adaptation of RPL for
  BLE would be difficult as it has a very high implementation complexity but perhaps
  performance gains would outweigh this complexity.
