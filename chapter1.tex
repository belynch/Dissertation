\chapter{Introduction}

	\section{Introduction}
	Attendance tracking is commonly used in places such as universities, to determine
	student attendance in lectures, and in organisations, for taking employee attendance
	and for assembly scenarios such as in fire evacuation procedures.

	Attendance tracking has mainly been achieved by performing manual roll calls using
	pen and paper, but has had implementations using technologies such as barcodes,
	Radio-frequency identification (RFID), and Near field communication (NFC).
	In the case of barcodes and NFC, these technologies require users to manually
	scan themselves for their attendance to be taken. In the case of RFID, tag collision
	can occur when numerous tags in the same area respond at the same time. Although
	implementations exist using these technologies, they have limitations and are
	not always effortless for the user.

	Bluetooth Low Energy (BLE) is a version of Bluetooth which was designed specifically
	with the Internet of Things (IoT) in mind. It has ultra-low peak, average, and idle
	power consumption and operates in the industrial, scientific, and medical (ISM)
	radio band (2.400 - 2.4835 GHz). BLE has two modes of communication, broadcasting
	and connections, with broadcasting most commonly being used to discover devices in order
	to establish connections.

	Microcontrollers with BLE radios, such as Nordic Semiconductor's nRF51 series
	devices, are extremely small (6mm x 6mm x 1mm) and relatively cheap (from 2 euro).
	It is feasible to incorporate these microcontrollers into student or organisation
	ID cards. With the addition of energy efficient applications with a focus on radio
	duty cycling, a microcontroller should be able to run on a coin cell battery
	for the lifecycle of the ID card.

	High crowd density is one of the main characteristics of scenarios where
	attendance tracking is commonly used. With the use of multi-hop routing,
	this high crowd density can be exploited to reduce the required transmission
	range of devices, which in turn reduces the transmission power required.

	The combination of a low power routing protocol, such as the Ad-hoc On Demand
	Distance Vector routing protocol, with the exclusive use of BLE broadcasting results
	in an implementation with small packet sizes, low packet processing times, and
	removes the overhead of connection establishment, creating an ideal framework
	for efficient and long lasting attendance tracking applications.

	\section{Problem Area}
	As mentioned, the attendance tracking is focused in areas of high crowd densities.
	Devices in the location where attendance tracking is taking place will form wireless,
	mobile ad-hoc networks (MANET).

	Attendance tracking in MANETs is a form of network member discovery and so one of
	the most important requirements in the selection of an appropriate routing protocol
	is its performance in the discovery of new routes. Once a route is established,
	member presence knowledge will have been acquired and the route will only be used
	in the establishment of other routes.

	Energy efficiency will be one of the main requirements for an attendance tracking
	application, as it will have to last for the lifetime of the user's ID card. For
	the purpose of this research we consider that lifetime to be a year. The main
	mechanism to achieve this energy efficiency is radio duty cycling. That is, the
	radio of the device will remain in a low power mode for the majority of its life,
	only entering a higher power mode when attendance is being taken.

	\section{Motivation for this research}
	The Internet of Things (IoT) has recently been in the spotlight as one of the fastest
	growing information technology sectors. According to a report \cite{IHS_IOT} from
	the financial services company IHS Markit, the number of IoT devices is expected
	to rise by fifteen percent year-over-year to reach twenty billion in 2017.

	According to the IoT Developer Survey 2017 \cite{SURVEY_IOT}, Bluetooth Low Energy
	has seen a growth of twelve percent in its use as a connectivity protocol for IoT solutions
	since 2015. In addition to this, the majority of major android branded
	smartphones (Android OS 4.3 and above), iOS devices (model 4S and onwards),
	and Windows based smart phones are BLE enabled. The early adoption of BLE in Apple's
	iBeacon has also aided in the wider adoption of BLE technology.

	The explosion of the IoT and the growing adoption rate of BLE make it an exciting
	technology to work with. It is still a relatively new technology and further
	research into its capabilities and possible applications is still required.

	\section{Objectives}
	The goals of this dissertation are as follows:
	\begin{itemize}
		\item The design of an attendance tracking protocol based on an existing routing
		protocol suitable for member identification in an ad-hoc network.
		\item The implementation of a module encapsulating the functionality of the
		chosen routing protocol.
		\item The implementation of an application which makes use of the above module
		to track attendance.
		\item An analysis of the performance of the protocol in BLE.
	\end{itemize}

	\section{Dissertation Outline}
	This dissertation begins with a literature review in Chapter 2 which provides
	an understanding of radio duty cycling, an overview of appropriate routing
	protocols, and a review of their performance. Chapter 3 provides a description of the key aspects of BLE for the
	development of applications, a discussion of design decisions related to the implementation
	of a routing protocol for BLE, as well as a description of two protocols
	for attendance tracking. Chapter 4 discusses the implementation of these protocols
	and provides some security considerations with regards to the implementation.
	Chapter 5 provides an overview of the evaluation methodology, results from medium
	scale testing, and an analysis of these results. Finally, Chapter 6 summarises the
	contributions made by this dissertation, a general discussion on the outcome of this thesis, and areas
	for future work.
