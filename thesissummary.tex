\begin{center}
  \textbf{\LARGE{Summary}}
\end{center}

This thesis aimed to design and implement an attendance tracking protocol
using Bluetooth Low Energy (BLE). The attendance tracking protocol would be implemented
in a BLE capable microcontroller which would be incorporated in an ID card along with
a coin cell battery.

Attendance tracking is performed in areas of high population densities such as
lecture theatres and evacuation assembly points. These are ideal network topologies for multi-hop
routing protocols, which can take advantage of the high density, reducing the required transmission
power, and consequently, power consumption of devices in the network.

The Ad-hoc on Demand Distance Vector (AODV) routing protocol is
chosen as a suitable protocol for attendance tracking after some analysis of existing
routing protocols and research investigating their performance. Two attendance tracking
protocols are designed, \textit{Roll Call} and \textit{Route-toZero}, that are based on AODV's route discovery mechanism.
This mechanism is implemented on Nordic Semiconductor's nRF5 series devices
using their nRF SDK. These two attendance tracking protocol designs are then implemented
on top of the AODV implementation.

The attendance tracking applications are evaluated in a physical network of ten nodes
comprised of nRF51 devices. Metrics including the node discovery rate, number of
packets sent and processed per node, and number of packets sent and processed per hop
are measured. The viability of the attendance tracking protocol
and AODV implementation are shown, with a best recorded node discovery rate of 1.34
nodes discovered per second.

Based on these results, a number of areas for further work are outlined. One of
these is the improvement of the AODV implementation to include route maintenance
to handle route errors. This would be required in larger networks and would
allow for mobility in nodes while tracking attendance. Another area for further work
is the implementation of these designs in a simulator in order to measure performance in larger
and denser networks. The final area is the implementation of other protocols which might
be suitable for attendance tracking.

\newpage
