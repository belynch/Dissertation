\begin{thesisabstract}

The possibilities to improve the digitisation of attendance tracking have grown
with the recent explosion of the Internet of Things. Technologies
such as Bluetooth Low Energy (BLE) allow for the development of applications that
can run on extremely small microcontrollers and can last for years on a
coin cell battery.

The aim of this thesis is to design an attendance tracking application based on
an existing low power ad-hoc routing protocol that is capable of lasting for the
lifetime of an ID card, which a microcontroller and batter could be incorporated into.

Two attendance tracking protocols are designed that are based on the Ad-hoc On Demand
Distance Vector (AODV) routing protocol's route discovery mechanism. This route
discovery mechanism is implemented on Nordic Semiconductor's BLE stack. Both of these attendance
tracking protocol designs are then built on top of this AODV implementation.

The attendance tracking application is evaluated in a physical network of ten nodes,
comprised of nRF51 devices. The viability of the attendance tracking protocol
and AODV implementation are shown, with a best recorded node discovery rate of 1.34
nodes discovered per second.

\end{thesisabstract}
